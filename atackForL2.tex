\documentclass[a4paper, 12pt]{report}
\usepackage[T2A]{fontenc}
\usepackage[english, russian]{babel}
\usepackage{graphicx}
\graphicspath{{./Images/}}
\usepackage[utf8]{inputenc}
\usepackage[backend=bibtex,bibencoding=utf8,sorting=nty,maxcitenames=2,style=numeric-comp]{biblatex}
\addbibresource{bibliography.bib}

\begin{document}
	\chapter{Атаки на сети уровня L2}
		\section{ARP-Spoofing}
		
		ARP-spoofing \cite{бирюков2022информационная} — разновидность сетевой атаки типа MITM, применяемая в сетях с использованием протокола ARP. В основном применяется в сетях Ethernet. Атака основана на недостатках протокола ARP.
		
		Злоумышленник выбирает машину или машины жертвы
		Первым шагом в планировании и реализации атаки ARP Spoofing является выбор цели. Это может быть конкретная конечная точка в сети, группа конечных точек или сетевое устройство, такое как маршрутизатор. Маршрутизаторы являются привлекательными целями, поскольку успешное отравление ARP маршрутизатора может нарушить трафик для всей подсети.
		Злоумышленник запускает инструменты и начинает атаку
		\begin{figure}[h!]
			\centering
			\includegraphics[scale=0.45]{img/2.jpg}
			\caption{MITM}
			\label{chargets}
		\end{figure}
		Всем злоумышленникам, желающим выполнить отравление ARP, легко доступен широкий спектр инструментов. После запуска выбранного инструмента и настройки соответствующих параметров злоумышленник начинает атаку. Он может незамедлительно начать рассылку сообщений ARP или дождаться получения запроса.
		Злоумышленник выполняет определенные действия с некорректно направленным трафиком
		После повреждения кэша ARP на устройстве (устройствах) жертвы злоумышленник обычно выполняет какие-то действия с некорректно направленным трафиком. Он может просматривать или изменять его, либо создать «черную дыру», чтобы данные никогда не доходили до адресата. Выбор действий зависит от мотивов злоумышленника. Пример реализация ARP-спуфинга на Python:
		%\documentclass{article}
		%\usepackage{listings}
		

		\begin{verbatim}
			import socket
			import time
			
			interface = "wlan0"  # Прослушиваемый сетевой интерфейс
			mac = b"\xbb\xbb\xbb\xbb\xbb\xbb"  # Наш MAC-адрес, он же bb:bb:bb:bb:bb:bb
			
			gateway_ip = socket.inet_aton("192.168.1.1")  # IP-адрес шлюза
			gateway_mac = b"\xaa\xaa\xaa\xaa\xaa\xaa"  # MAC-адрес шлюза
			
			victim_ip = socket.inet_aton("192.168.1.2")  # IP-адрес жертвы
			victim_mac = b"\xcc\xcc\xcc\xcc\xcc\xcc"  # MAC-адрес жертвы
			
			connect = socket.socket(socket.PF_PACKET, socket.SOCK_RAW, socket.htons(0x0800))
			connect.bind((interface, socket.htons(0x0800)));
			}
		\end{verbatim}
		
	 \printbibliography
	
\end{document}
